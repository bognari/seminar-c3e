% Offizielle Beispieldatei für beamer-Vorlage aus tubslatex Version 0.3beta2
\documentclass[fleqn,11pt,aspectratio=43]{beamer}

\usepackage[ngerman]{babel}
\usepackage[utf8x]{inputenc}
\usepackage{graphicx}
\usepackage{listings}
\PassOptionsToPackage{svgnames}{xcolor}
\usepackage{color}
\usepackage{pxfonts}

%\usepackage{DejaVuSansMono}
\usetheme[%
  %cmyk,%<cmyk/rgbprint>,          Auswahl des Farbmodells
  orange,%<blue/orange/green/violet> Auswahl des Sekundärfarbklangs
  %dark,%<light,medium>        Auswahl der Helligkeit
  %colorhead,%    Farbig hinterlegte Kopfleiste
  %colorfoot,%    Farbig hinterlegt Fußleiste auf Titelseite
  colorblocks,%   Blöcke Farbig hinterlegen
  %nopagenum,%    Keine Seitennumer in Fußzeile
  %nodate,%       Kein Datum in Fußleiste
  %tocinheader,%   Inhaltsverzeichnis in Kopfleiste
  %tinytocinheader,% kleines Kopfleisten-Inhaltsverzeichnis
  %widetoc,%      breites Kopfleisten-Inhaltsverzeichnis
  %narrowtoc,%    schmales Kopfleisten-Inhaltsverzeichnis
  %nosubsectionsinheader,%  Keine subsections im Kopfleisten-Inhaltsverzeichnis
  %nologoinfoot,% Kein Logo im Fußbereich darstellen
  ]{tubs}

% Titelgrafik, automatisch beschnitten, Weitere Optionen: <scaled/cropx/cropy>
%\titlegraphic[scaled]{\includegraphics{img/titlepicture.jpg}}
\titlegraphic[cropy]{\includegraphics{img/zynq.jpg}}

% Logo, dass auf Titelseiten oben rechts und auf Inthaltsseiten unten rechts
% dargestellt wird. Es wird jeweils automatisch skliert
\logo{\includegraphics{img/logo_mit_text.pdf}}

% settings for source code highlighting
\lstset{ %
  backgroundcolor=\color{white},         % choose the background color
  %basicstyle=\ttfamily\footnotesize,     % size of fonts used for the code
  basicstyle=\ttfamily\footnotesize,     % size of fonts used for the code
  breaklines=true,                       % automatic line breaking only at whitespace
  captionpos=b,                          % sets the caption-position to bottom
  commentstyle=\color{tuGreenDark80},    % comment style
  escapeinside={\%*}{*)},                % if you want to add LaTeX within your code
  keywordstyle=\color{tuBlueMedium100}\bfseries, % keyword style
  stringstyle=\color{tuVioletMedium},    % string literal style
}

\AtBeginSection[]{
  \begin{frame}[noframenumbering] 
  		\scriptsize
  		\frametitle{Überblick}  
  		\tableofcontents[currentsection, subsectionstyle=show]
  \end{frame}
}

\AtBeginSubsection[]{
  \begin{frame}[noframenumbering]
    	\scriptsize 
  		\frametitle{\insertsectionhead - \insertsubsectionhead} 
  		\tableofcontents[ 
  			currentsubsection, 
  		    sectionstyle=show/hide, 
  		   	subsectionstyle=show/shaded/hide] 
  \end{frame}
}

\usepackage{tikz}
\usetikzlibrary{decorations.markings}

\pgfdeclarelayer{edgelayer}
\pgfdeclarelayer{nodelayer}
\pgfsetlayers{edgelayer,nodelayer,main}

\tikzstyle{none}=[inner sep=0pt]
\definecolor{hexcolor0x0074ff}{rgb}{0.000,0.455,1.000}
\definecolor{hexcolor0xff2615}{rgb}{1.000,0.149,0.082}

\definecolor{myblack}{rgb}{0.000,0.000,0.000}
\definecolor{mywhite}{rgb}{1.000,1.000,1.000}

\tikzstyle{setA}=[circle,fill=hexcolor0x0074ff,draw=myblack]
\tikzstyle{setB}=[circle,fill=hexcolor0xff2615,draw=myblack]
\tikzstyle{node}=[circle,fill=mywhite,draw=myblack,scale=.1]

% Titelseite
\title{Seminar Technische Informatik}
\subtitle{Top 10 algorithms in data mining}
\author{Stephan Mielke}
\date{22.01.2015}

\begin{document}

\begin{frame}[plain, noframenumbering]
\titlepage
\end{frame}

\begin{frame}{Inhalt}
\tableofcontents
\end{frame}

\section*{Einleitung~}

\begin{frame}[plain]{\insertsectionhead - Der Weltraum unendliche Weiten \dots}

\begin{figure}
\includegraphics[scale=0.6,trim={40 400 40 80},clip]{hs-2004-07-a-pdf}
\caption{Hubble Ultra Deep Field\cite{HUDF}}
\label{fig:hs-2004-07-a-pdf}
\end{figure}
\end{frame}

\begin{frame}{\insertsectionhead - Einsatz von DM in der Astronomie}
\begin{itemize}
\item Klassifizierung von Sternen mit Knn\footnote{k-nearest neighbor}
\item Manuelle Klassifizierung unmöglich \cite{ester2000knowledge}
\item Pro Bild mehre $10000$ Objekte
\item Kepler z.B. hat 13.2m Objekte erkannt
\item Benutzung von Klassifizierungsalgorithmen aus DM
\only<1>{\item Je Objekt 9 Attribute (8 Isophotenformen, Leuchtkraft)
\item Ausgabewert \glqq stellary\grqq 
\begin{itemize}
\item $0.0 - 0.1$ Galaxie
\item $0.9 - 1.0$ Stern
\end{itemize}} % zahlen nennen
\end{itemize}
\only<2>{
\begin{table}
\begin{tabular}{l|l}
Name & Erkennung\\ \hline
Random Forest & $82,89\%$ \\
Decision Tree & $80,68\%$ \\
Artificial Neural Network & $75.82\%$ \\
Support Vector Machines & $37,82\%$ \\
\end{tabular}
\caption{Erkennungsraten der Algorithmen Stern / Galaxie\cite{o2009star}}
\end{table}}
\end{frame}

\section{Data Mining~}

\begin{frame}{\insertsectionhead - Einleitung\cite{ester2000knowledge}}
\begin{itemize}
\item Gehört zum Gebiet des KDD (Knowledge Discovery in Databases)
\item Idee: \emph{Wissen} durch \emph{Daten}
\item Einsatz in der Forschung, Vermarktung, Medizin, (Wetter)-Vorhersagen, 
Betrugsaufklärung usw.
\end{itemize}
\begin{block}{Definition nach Fayyad\cite{fayyad1996data}}
Knowledge Discovery in Databases describes the non-trivial process of 
identifying valid, novel, potentially useful, and ultimately understandable 
patterns in data.
\end{block}
\end{frame}

\begin{frame}{\insertsectionhead - Einordnung}
\begin{figure}
\includegraphics[page=5,scale=0.63,trim={48 500 55 80},clip]{dm_fayyad.pdf}
\caption{KDD nach Fayyad\cite{fayyad1996data}}
\label{fig:fayyad1996data}
\end{figure}
\end{frame}

%\begin{frame}{\insertsectionhead - Algorithmen Kategorien}
%\begin{block}{}
%\centering
%\alert<1>{There are known knowns,} [...]\\
%\alert<2>{There are known unknowns}, [...]\\
%\alert<3>{There are also unknown unknowns}\footnote{Donald H. Rumsfeld, US 
%Secretary of 
%Defense at a US Department of Defense news briefing, 12 February 
%2002}\cite{AstronomicalDM}
%\end{block}
%
%\begin{alertblock}{}
%\centering
%\only<1>{Klassifikation}
%\only<2>{Regression}
%\only<3>{Clustering}
%\only<4>{Assoziation}
%\end{alertblock}
%\end{frame}


\subsection{Top 10 algorithms in data mining}
\begin{frame}{\insertsectionhead - \insertsubsectionhead\cite{wu2008top}}
\begin{itemize}
\item Anlass: IEEE International Conference on Data Mining
\item Datum: Dezember 2006
\item Erstellung: Jeder ACM KDD Innovation Award oder IEEE ICDM Research 
Contributions Award Preisträger nominierte 10 Algorithmen
\item Nur Nominierte mit $\geq$ 50 Referenzierungen in \emph{Google Scholar}
\item 
{\small\url{http://www.cs.uvm.edu/~icdm/algorithms/CandidateList.shtml}}
\item Per Abstimmung finden der Top 10
\item Das Paper: Top 10 algorithms in data mining \cite{wu2008top}
\end{itemize}
\end{frame}

\begin{frame}{\insertsectionhead - \insertsubsectionhead\cite{wu2008top}}
\begin{columns}[onlytextwidth]
    \column{0.5\textwidth}
		\begin{enumerate}
		\item C4.5 und ähnliche
		\item \textbf{k-means}
		\item \textbf{Suport Vector Machines}
		\item \textbf{Apriori}
		\item EM Algorithm
		\end{enumerate}
    \column{0.5\textwidth}
	    \begin{enumerate}
	    \setcounter{enumi}{5}
	    \item PageRank
	    \item AdaBoost
	    \item k-nearest neighbor
	    \item Naive Bayes
	    \item CART
	    \end{enumerate}
\end{columns}
\end{frame}

%\begin{frame}{\insertsectionhead - \insertsubsectionhead}
%
%\end{frame}

\subsection{Clustering~}

\begin{frame}{\insertsectionhead - \insertsubsectionhead - Einleitung\cite{ester2000knowledge}}
\begin{itemize}
\item Einordnung von Objekten in unbekannten Klassen
\item Finden der Funktion die Objekte gruppiert
%\item Cluster haben unterschiedliche Formen, sind verschachtelt usw.
\item Ähnlichkeit von Objekten durch eine Distanzfunktion ermitteln
\end{itemize}
\end{frame}

\begin{frame}{\insertsectionhead - \insertsubsectionhead - Cluster\cite{dwh}}
\begin{itemize}
\item Formen: sehr unterschiedlich
\item Flach oder Hierarchisch
\item Anzahl von Clustern:
\begin{itemize}
\item Festgelegte Anzahl von $k$-Clustern
\item Anzahl hängt von der Qualitätsgüte der Cluster ab
\end{itemize} 
\item Qualitätsgüte: nicht zu klein oder groß
\item Hard oder Soft - Clustering
\item Keine großen \glqq Lücken\grqq\ zwischen den Daten
\item Cluster durch Heuristiken sonst zu großer Aufwand
\end{itemize}
\end{frame}

\begin{frame}{\insertsectionhead - \insertsubsectionhead - Distanzfunktion\cite{ester2000knowledge}}
\begin{itemize}
\item Menge von Objekten $O = \{o_1, o_2, \ldots, o_n\}$
\item Jedes Objekt hat $A_i$ Aattribute
\item Attributarten:
\begin{itemize}
\item Kategorische Attribute
\item Nummerische Attribute
\end{itemize}
\item Es muss gelten 1.-3., für Metrik 4.:
\begin{align}
dist(o_1, o_2) &= d \in R^{n\geq 0}\\
dist(o_1, o_2) &= 0 \mbox{ genau dann wenn } o_1 = o_2\\
dist(o_1, o_2) &= dist(o_2, o_1) \mbox{ (Symmetrie)}\\
dist(o_1, o_3) &\leq dist(o_1, o_2) + dist(o_2, o_3)
\end{align}
\item Manchmal auch Ähnlichkeitsfunktion genannt $\Rightarrow$ Interpretation anders herum.
\end{itemize}
\end{frame}

\begin{frame}[fragile]{\insertsectionhead - \insertsubsectionhead - Distanzfunktion\cite{ester2000knowledge}}
\begin{itemize}
\item Datensätze $x = (x_1, \ldots, x_n)$ mit nummerischen Attributen $x_i$
\only<1>{\begin{itemize}
\item Euklidische-Distanz: $dist(x,y) = \sqrt{(x_1 - y_1)^2 + \ldots + (x_n - y_n)^2}$
\item Manhattan-Distanz: ~$dist(x,y) = |x_1 - y_1| + \ldots + |x_n - y_n|$
\item Maximum-Metrik: ~~~~~$dist(x,y) = \mbox{max}\big(|x_1 - y_1| + \ldots + |x_n - y_n|\big)$
\item Alg. $L_p$-Metrik: ~~~~~~~~~~$dist(x,y) = \sqrt[p]{\sum\limits_{i=1}^{d}{(x_i-y_i)^p}}$
\end{itemize}}
\item Datensätze $x = (x_1, \ldots, x_n)$ mit kategorischen Attributen $x_i$
\only<1>{\begin{itemize}
\item Summe der Unterschiede
\item $dist(x,y) = \sum\limits_{i=1}^{a}\delta(x_i, y_i)$
\item $\delta(x_i, y_i) = \begin{cases}
0 \mbox{ wenn } (x_i = y_i)\\
1 \mbox{ wenn } (x_i \neq y_i)
\end{cases} $
\end{itemize}}
\item Endliche Mengen $x = \{x_1, \ldots, x_n\}$\\
\only<1>{
Anteil verschiedener: $dist(x,y) = \frac{|x\cup y| - |x \cap y|}{|x \cup y|}$
}
\end{itemize}
\end{frame}

\begin{frame}{\insertsectionhead - \insertsubsectionhead - Beispiel\cite{ester2000knowledge}}
\begin{itemize}
\item Clustering von Web-Sessions zur Bestimmung von Benutzergruppen
\item Datenquelle: Logfile eines Webservers
\item Eintrag: IP, User-ID, Timestamp, URL, \ldots
\item Einträge werden nach Session gruppiert, nach einem Zeitfenster
\item Session: IP, User-ID, Liste von URLs
\item URLs werden geclustert, z.B.: Distanzfunktion für endliche Mengen
\item Wissen:
\begin{itemize}
\item Benutzergruppen / Benutzerprofilen, für Marketingstrategien 
\item URLs sind durch Interessen verbunden, Optimierung für Zugriffsgewohnheiten 
\end{itemize}
\item Ein Sozialmediabutton kann auch die nötigen Informationen liefern.
\end{itemize}
\end{frame}


\subsubsection{k-means}\label{kmeans}

\begin{frame}{\insertsectionhead - \insertsubsectionhead - $k$-means\cite{dwh}}
\begin{itemize}
\item Hartes Flaches Clustering
\item Bekannte Anzahl von $k$ Clustern
\item Daten als Vektoren
\item Idee: Minimiert den Abstand vom Clusterzentrum zu den Daten
\item Cluster ist Definiert als:
\begin{itemize}
\item $A = \{d_l, \ldots, d_m\}$, $A$ ist ein Cluster und $d_i$ Element 
\item Zentrum ist: $\mu(A) = \frac{l}{m}\sum\limits_{i=l}^{m}{d_i}$
\end{itemize}
\item Qualität: gut wenn RSS$(\ldots)$ minimal ist
\begin{itemize}
\item Cluster: ~$\mbox{RSS}\,(A) = \sum\limits_{i=l}^{m}\big\|d_i - \mu(A)\big\|^2$
\item Gesamt: $\mbox{RSS}\,(A_l, \ldots, A_k) = \sum\limits_{j=l}^{k}\,\mbox{RSS}\,(A_j)$
\end{itemize}
\end{itemize}
\end{frame}

\begin{frame}{\insertsectionhead - \insertsubsectionhead - $k$-means\cite{dwh}}
Der $k$-means Algorithmus (Lloyd's Algorithmus)
\begin{enumerate}
\item Selektiere zufällig $k$ Schwerpunkte als Startwert
\item Erstelle $k$ leere Cluster
\item Weise jedem Cluser einen Schwerpunkt zu
\item Weise jedem Datenvektor den den Cluster mit dem nächstem Schwerpunkt zu
\item Berechne den Schwerpunkt jedes Clusters neu
\item Teste ob die Qualität des Clusterings ausreicht, sonst gehe zu 2.
\end{enumerate}
\end{frame}

\begin{frame}{\insertsectionhead - \insertsubsectionhead - $k$-means}
\begin{figure}
\scalebox{1.1}{\tikzstyle{none}=[inner sep=0pt]
\definecolor{hexcolor0x0074ff}{rgb}{0.000,0.455,1.000}
\definecolor{hexcolor0xff2615}{rgb}{1.000,0.149,0.082}

\definecolor{myblack}{rgb}{0.000,0.000,0.000}
\definecolor{mywhite}{rgb}{1.000,1.000,1.000}

\tikzstyle{sp}=[circle,fill=myblack,draw=myblack, scale=1]
\tikzstyle{setA}=[circle,fill=hexcolor0x0074ff,draw=myblack]
\tikzstyle{setB}=[circle,fill=hexcolor0xff2615,draw=myblack]
\tikzstyle{setC}=[circle,fill=mywhite,draw=myblack]
\tikzstyle{node}=[circle,fill=mywhite,draw=myblack,scale=.1]


\begin{tikzpicture}
	%\begin{pgfonlayer}{nodelayer}
		\node [style=setC] (0) at (-1.75, 1.75) {};
		\node [style=setC] (1) at (-3, 0.5) {};
		\node [style=setC] (2) at (0,2) {};
		\node [style=setC] (3) at (-1.25, 0.75) {};
		\node [style=setC] (4) at (-2.25, 1) {};
		\node [style=setC] (5) at (-3.25, 1.75) {};
		\node [style=setC] (6) at (-1.75, 2.75) {};
		\node [style=setC] (7) at (-1, 1.5) {};
		\node [style=setC] (8) at (2.25, 1.5) {};
		\node [style=setC] (9) at (1.25, 0.75) {};
		\node [style=setC] (10) at (2.75, -0.25) {};
		\node [style=setC] (11) at (0,-0.5) {};
		\node [style=setC] (12) at (3.25, 1) {};
		\node [style=setC] (13) at (2, 0.5) {};
		\node [style=setC] (14) at (2.5, -1.5) {};
		\node [style=setC] (15) at (-1,-1.5) {};
		\node [style=setC] (16) at (1.25, -1.75) {};
		\node [style=setC] (17) at (1.75, -0.25) {};
		\node [style=setC] (18) at (-2.25, -0.25) {};
		\node [style=setC] (19) at (1.75, 1) {};

\node [sp] at (-1,-0.5) {};
\node [sp] at (1,1.5) {};
\end{tikzpicture}}
\caption{Ersten 3 Phasen, $k = 2$}
\end{figure}
\end{frame}

\begin{frame}{\insertsectionhead - \insertsubsectionhead - $k$-means}
\begin{figure}
\scalebox{1.1}{\tikzstyle{none}=[inner sep=0pt]
\definecolor{hexcolor0x0074ff}{rgb}{0.000,0.455,1.000}
\definecolor{hexcolor0xff2615}{rgb}{1.000,0.149,0.082}

\definecolor{myblack}{rgb}{0.000,0.000,0.000}
\definecolor{mywhite}{rgb}{1.000,1.000,1.000}

\tikzstyle{sp}=[circle,fill=myblack,draw=myblack, scale=1]
\tikzstyle{setA}=[circle,fill=hexcolor0x0074ff,draw=myblack]
\tikzstyle{setB}=[circle,fill=hexcolor0xff2615,draw=myblack]
\tikzstyle{setC}=[circle,fill=mywhite,draw=myblack]
\tikzstyle{node}=[circle,fill=mywhite,draw=myblack,scale=.1]


\begin{tikzpicture}
		\node [style=setA] (0) at (-1.75, 1.75) {};
		\node [style=setB] (1) at (-3, 0.5) {};
		\node [style=setA] (2) at (0,2) {};
		\node [style=setB] (3) at (-1.25, 0.75) {};
		\node [style=setB] (4) at (-2.25, 1) {};
		\node [style=setB] (5) at (-3.25, 1.75) {};
		\node [style=setA] (6) at (-1.75, 2.75) {};
		\node [style=setA] (7) at (-1, 1.5) {};
		\node [style=setA] (8) at (2.25, 1.5) {};
		\node [style=setA] (9) at (1.25, 0.75) {};
		\node [style=setA] (10) at (2.75, -0.25) {};
		\node [style=setB] (11) at (0,-0.5) {};
		\node [style=setA] (12) at (3.25, 1) {};
		\node [style=setA] (13) at (2, 0.5) {};
		\node [style=setB] (14) at (2.5, -1.5) {};
		\node [style=setB] (15) at (-1,-1.5) {};
		\node [style=setB] (16) at (1.25, -1.75) {};
		\node [style=setA] (17) at (1.75, -0.25) {};
		\node [style=setB] (18) at (-2.25, -0.25) {};
		\node [style=setA] (19) at (1.75, 1) {};

\node [sp] (v1) at (-1,-0.5) {};
\node [sp] (v2) at (1,1.5) {};
\draw  (3) edge (v1);
\draw  (11) edge (v1);
\draw  (15) edge (v1);
\draw  (18) edge (v1);
\draw  (4) edge (v1);
\draw  (1) edge (v1);
\draw  (5) edge (v1);
\draw  (16) edge (v1);
\draw  (14) edge (v1);
\draw  (6) edge (v2);
\draw  (0) edge (v2);
\draw  (7) edge (v2);
\draw  (2) edge (v2);
\draw  (9) edge (v2);
\draw  (19) edge (v2);
\draw  (8) edge (v2);
\draw  (12) edge (v2);
\draw  (13) edge (v2);
\draw  (17) edge (v2);
\draw  (10) edge (v2);
\end{tikzpicture}}
\caption{Phase 4, Zuordnung nur beispielhaft}
\end{figure}
\end{frame}

\begin{frame}{\insertsectionhead - \insertsubsectionhead - $k$-means}
\begin{figure}
\scalebox{1.1}{\tikzstyle{none}=[inner sep=0pt]
\definecolor{hexcolor0x0074ff}{rgb}{0.000,0.455,1.000}
\definecolor{hexcolor0xff2615}{rgb}{1.000,0.149,0.082}

\definecolor{myblack}{rgb}{0.000,0.000,0.000}
\definecolor{mywhite}{rgb}{1.000,1.000,1.000}

\tikzstyle{sp}=[circle,fill=myblack,draw=myblack, scale=1]
\tikzstyle{setA}=[circle,fill=hexcolor0x0074ff,draw=myblack]
\tikzstyle{setB}=[circle,fill=hexcolor0xff2615,draw=myblack]
\tikzstyle{setC}=[circle,fill=mywhite,draw=myblack]
\tikzstyle{node}=[circle,fill=mywhite,draw=myblack,scale=.1]


\begin{tikzpicture}
		\node [style=setA] (0) at (-1.75, 1.75) {};
		\node [style=setB] (1) at (-3, 0.5) {};
		\node [style=setA] (2) at (0,2) {};
		\node [style=setB] (3) at (-1.25, 0.75) {};
		\node [style=setB] (4) at (-2.25, 1) {};
		\node [style=setB] (5) at (-3.25, 1.75) {};
		\node [style=setA] (6) at (-1.75, 2.75) {};
		\node [style=setA] (7) at (-1, 1.5) {};
		\node [style=setA] (8) at (2.25, 1.5) {};
		\node [style=setA] (9) at (1.25, 0.75) {};
		\node [style=setA] (10) at (2.75, -0.25) {};
		\node [style=setB] (11) at (0,-0.5) {};
		\node [style=setA] (12) at (3.25, 1) {};
		\node [style=setA] (13) at (2, 0.5) {};
		\node [style=setB] (14) at (2.5, -1.5) {};
		\node [style=setB] (15) at (-1,-1.5) {};
		\node [style=setB] (16) at (1.25, -1.75) {};
		\node [style=setA] (17) at (1.75, -0.25) {};
		\node [style=setB] (18) at (-2.25, -0.25) {};
		\node [style=setA] (19) at (1.75, 1) {};

\node [none] (v2) at (-1,-0.5) {};
\node [none] (v3) at (1,1.5) {};

\node [sp] (v1) at (-1.5,0) {};
\node [sp] (v4) at (1.5,1) {};
\draw  (v2) edge (v1);
\draw  (v3) edge (v4);
\end{tikzpicture}}
\caption{Phase 5, Schwerpunkte sind nur beispielhaft}
\end{figure}
\end{frame}

\begin{frame}{\insertsectionhead - \insertsubsectionhead - $k$-means}
\begin{figure}
\scalebox{1.1}{\tikzstyle{none}=[inner sep=0pt]
\definecolor{hexcolor0x0074ff}{rgb}{0.000,0.455,1.000}
\definecolor{hexcolor0xff2615}{rgb}{1.000,0.149,0.082}

\definecolor{myblack}{rgb}{0.000,0.000,0.000}
\definecolor{mywhite}{rgb}{1.000,1.000,1.000}

\tikzstyle{sp}=[circle,fill=myblack,draw=myblack, scale=1]
\tikzstyle{setA}=[circle,fill=hexcolor0x0074ff,draw=myblack]
\tikzstyle{setB}=[circle,fill=hexcolor0xff2615,draw=myblack]
\tikzstyle{setC}=[circle,fill=mywhite,draw=myblack]
\tikzstyle{node}=[circle,fill=mywhite,draw=myblack,scale=.1]

\begin{tikzpicture}
		\node [style=setB] (0) at (-1.75, 1.75) {};
		\node [style=setB] (1) at (-3, 0.5) {};
		\node [style=setA] (2) at (0,2) {};
		\node [style=setB] (3) at (-1.25, 0.75) {};
		\node [style=setB] (4) at (-2.25, 1) {};
		\node [style=setB] (5) at (-3.25, 1.75) {};
		\node [style=setB] (6) at (-1.75, 2.75) {};
		\node [style=setB] (7) at (-1, 1.5) {};
		\node [style=setA] (8) at (2.25, 1.5) {};
		\node [style=setA] (9) at (1.25, 0.75) {};
		\node [style=setA] (10) at (2.75, -0.25) {};
		\node [style=setB] (11) at (0,-0.5) {};
		\node [style=setA] (12) at (3.25, 1) {};
		\node [style=setA] (13) at (2, 0.5) {};
		\node [style=setA] (14) at (2.5, -1.5) {};
		\node [style=setB] (15) at (-1,-1.5) {};
		\node [style=setA] (16) at (1.25, -1.75) {};
		\node [style=setA] (17) at (1.75, -0.25) {};
		\node [style=setB] (18) at (-2.25, -0.25) {};
		\node [style=setA] (19) at (1.75, 1) {};
		
\node [sp] (v1) at (-1.5,0) {};
\node [sp] (v4) at (1.5,1) {};

\draw  (v1) edge (3);
\draw  (7) edge (v1);
\draw  (0) edge (v1);
\draw  (4) edge (v1);
\draw  (1) edge (v1);
\draw  (5) edge (v1);
\draw  (6) edge (v1);
\draw  (18) edge (v1);
\draw  (15) edge (v1);
\draw  (11) edge (v1);
\draw  (2) edge (v4);
\draw  (9) edge (v4);
\draw  (19) edge (v4);
\draw  (8) edge (v4);
\draw  (12) edge (v4);
\draw  (13) edge (v4);
\draw  (10) edge (v4);
\draw  (17) edge (v4);
\draw  (16) edge (v4);
\draw  (14) edge (v4);
\end{tikzpicture}}
\caption{Phase 6 und noch mal von Phase 2 an}
\end{figure}
\end{frame}

\subsection{Klassifikation~}

\begin{frame}{\insertsectionhead - \insertsubsectionhead - Einleitung\cite{ester2000knowledge}}
\begin{itemize}
\item Einordnung von Objekten in bekannten Klassen
\item Trainingsdaten für Klassen $\Rightarrow$ Klassen bekannt
\item Finden der Funktion die Objekte möglichst genau zuordnet
\item Teilaufgaben:
\begin{itemize}
\item Zuordnung zu einer Klasse
\item Generierung von Wissen 
\end{itemize}
\end{itemize}
\end{frame}

\begin{frame}{\insertsectionhead - \insertsubsectionhead - Training\cite{ester2000knowledge}}
\begin{itemize}
\item Menge von Objekten $O = \{o_1, o_2, \ldots, o_n\}$
\item Klasse $c_i \in C = \{c_1, c_2, \ldots, c_n\}$ für jedes Objekt ist Bekannt
\item Jedes Objekt hat $A_i$ Klassifizierung-Aattribute
\item Attributarten:
\begin{itemize}
\item Kategorische Attribute
\item Nummerische Attribute
\end{itemize}
\end{itemize}
\end{frame}

\begin{frame}[fragile]{\insertsectionhead - \insertsubsectionhead - Beispiel\cite{ester2000knowledge}}
Trainingsdaten:
\begin{table}
\begin{tabular}{|p{2cm}|p{2cm}|p{2cm}|p{2cm}|}\hline
ID 	& Alter	& Autotyp	& Risikoklasse \\\hline \hline
1	& 23	& Familie	& Hoch\\
2	& 17	& Sport		& Hoch\\
3	& 43	& Sport		& Hoch\\
4	& 68	& Familie	& Niedrig\\
5	& 32	& LKW		& Niedrig \\\hline
\end{tabular}
\caption{Beispiele aus dem Buch\cite{ester2000knowledge}}
\end{table}
\pause
Das gesuchte Wissen
\begin{lstlisting}[language=Haskell]
if Alter > 50                    then Risikoklasse = Niedrig
if Alter <= 50 and Autotyp = LKW then Risikoklasse = Niedrig
                                 else Risikoklasse = Hoch
\end{lstlisting}
\end{frame}

\begin{frame}{\insertsectionhead - \insertsubsectionhead - Gesuchtes Wissen\cite{ester2000knowledge}}
\begin{itemize}
\item Formen:
\begin{itemize}
\item Entscheidungsbaum % geordnet nach Informationsgewinn
\item Funktion % alles mögliche, 
\item Vektor im Koordinatensystem
\end{itemize}
\item Anwendung immer dann, wenn die Klassen bekannt ist
\begin{itemize}
\item Unterscheidung von Stern / Galaxie
\item Sterne Einordnen
\item Zuordnung von Risikogruppen
\item Medizinforschung
\item \dots
\end{itemize}
\end{itemize}
\end{frame}

\subsubsection{Support vector machines}\label{svm}

\begin{frame}{\insertsectionhead - \insertsubsectionhead - SVM\cite{dwh}}
\begin{itemize}
\item Annahmen:
\begin{itemize}
\item Nur zwei Klassen
\item Jedes Objekt ist ein Vektor im Koordinatensystem
\end{itemize}
\item Ziel:  Hyperplane\footnote{Hyperebene} die den Raum teilt 
\item Training: Hyperplane mit maximalem Abstand zu allen Trainingsvektoren
\item Training: Hyperplane Begrenzungsobjekte sind Supportvektoren
\item Differenzfunktion $\delta(o_1, o_2)$ ist ähnlich zum Clustering
\end{itemize}
\end{frame}

\begin{frame}{\insertsectionhead - \insertsubsectionhead - SVM}
%Bild bei der die richtige Teilung gesucht wird (F 55)
\begin{figure}
\scalebox{.7}{\tikzstyle{none}=[inner sep=0pt]
\definecolor{hexcolor0x0074ff}{rgb}{0.000,0.455,1.000}
\definecolor{hexcolor0xff2615}{rgb}{1.000,0.149,0.082}

\definecolor{myblack}{rgb}{0.000,0.000,0.000}
\definecolor{mywhite}{rgb}{1.000,1.000,1.000}

\tikzstyle{setA}=[circle,fill=hexcolor0x0074ff,draw=myblack]
\tikzstyle{setB}=[circle,fill=hexcolor0xff2615,draw=myblack]
\tikzstyle{node}=[circle,fill=mywhite,draw=myblack,scale=.1]


\begin{tikzpicture}
	%\begin{pgfonlayer}{nodelayer}
		\node [style=setA] (0) at (-1.75, 1.75) {};
		\node [style=setA] (1) at (-3, 0.5) {};
		\node [style=setA] (2) at (0,2) {};
		\node [style=setA] (3) at (-1.25, 0.75) {};
		\node [style=setA] (4) at (-2.25, 1) {};
		\node [style=setA] (5) at (-3.25, 1.75) {};
		\node [style=setA] (6) at (-1.75, 2.75) {};
		\node [style=setA] (7) at (-1, 1.5) {};
		\node [style=setB] (8) at (2.25, 1.5) {};
		\node [style=setB] (9) at (1.25, 0.75) {};
		\node [style=setB] (10) at (2.75, -0.25) {};
		\node [style=setB] (11) at (0,-0.5) {};
		\node [style=setB] (12) at (3.25, 1) {};
		\node [style=setB] (13) at (2, 0.5) {};
		\node [style=setB] (14) at (2.5, -1.5) {};
		\node [style=setB] (15) at (-1,-1.5) {};
		\node [style=setB] (16) at (1.25, -1.75) {};
		\node [style=setB] (17) at (1.75, -0.25) {};
		\node [style=setA] (18) at (-2.25, -0.25) {};
		\node [style=setB] (19) at (1.75, 1) {};
		\node [style=node] (20) at (1,3.5) {};
		\node [style=node] (21) at (-2,-3.5) {};
		\node [style=node] (22) at (-4.25, -1.5) {};
		\node [style=node] (23) at (4.25, 2.75) {};
		\node [style=node] (24) at (2.5, 3.5) {};
		\node [style=node] (25) at (-4.25, -3.5) {};
	%\end{pgfonlayer}
	%\begin{pgfonlayer}{edgelayer}
		\draw [ultra thick] (22) to (23);
		\draw [ultra thick] (20) to (21);
		\draw [ultra thick] (25) to (24);
	%\end{pgfonlayer}
\node [none] (v1) at (-3.5,3) {};
\node [none] (v2) at (-5,-3.5) {};
\node [none] (v3) at (5,4) {};
\node [none] (v4) at (2,-3.5) {};
\draw [ultra thick] (v1) edge (v2);
\draw [ultra thick] (v3) edge (v4);
\end{tikzpicture}}
\caption{Gesucht: die richtige Hyperplane}
\end{figure}
\end{frame}

\begin{frame}{\insertsectionhead - \insertsubsectionhead - SVM}
\begin{figure}
\scalebox{.8}{\input{img/svm2.tikz}}
\caption{Gefunden: die richtige Hyperplane}
\end{figure}
\end{frame}

\begin{frame}{\insertsectionhead - \insertsubsectionhead - SVM}
\begin{figure}
\scalebox{.8}{\tikzstyle{none}=[inner sep=0pt]
\definecolor{hexcolor0x0074ff}{rgb}{0.000,0.455,1.000}
\definecolor{hexcolor0xff2615}{rgb}{1.000,0.149,0.082}

\definecolor{myblack}{rgb}{0.000,0.000,0.000}
\definecolor{mywhite}{rgb}{1.000,1.000,1.000}

\tikzstyle{setA}=[circle,fill=hexcolor0x0074ff,draw=myblack]
\tikzstyle{setB}=[circle,fill=hexcolor0xff2615,draw=myblack]
\tikzstyle{setc}=[circle,fill=mywhite,draw=myblack]
\tikzstyle{node}=[circle,fill=mywhite,draw=myblack,scale=.1]


\begin{tikzpicture}
	%\begin{pgfonlayer}{nodelayer}
		\node [style=setA] (0) at (-1.75, 1.75) {};
		\node [style=setA] (1) at (-3, 0.5) {};
		\node [style=setA] (2) at (0,2) {};
		\node [style=setA] (3) at (-1.25, 0.75) {};
		\node [style=setA] (4) at (-2.25, 1) {};
		\node [style=setA] (5) at (-3.25, 1.75) {};
		\node [style=setA] (6) at (-1.75, 2.75) {};
		\node [style=setA] (7) at (-1, 1.5) {};
		\node [style=setB] (8) at (2.25, 1.5) {};
		\node [style=setB] (9) at (1.25, 0.75) {};
		\node [style=setB] (10) at (2.75, -0.25) {};
		\node [style=setB] (11) at (0,-0.5) {};
		\node [style=setB] (12) at (3.25, 1) {};
		\node [style=setB] (13) at (2, 0.5) {};
		\node [style=setB] (14) at (2.5, -1.5) {};
		\node [style=setB] (15) at (-1,-1.5) {};
		\node [style=setB] (16) at (1.25, -1.75) {};
		\node [style=setB] (17) at (1.75, -0.25) {};
		\node [style=setA] (18) at (-2.25, -0.25) {};
		\node [style=setB] (19) at (1.75, 1) {};
		\node [style=node] (20) at (1.5, 3.5) {};
		\node [style=node] (21) at (-4.5,-3) {};
		\node [style=node] (22) at (3.5, 3.5) {};
		\node [style=node] (23) at (-2.5,-3) {};
		\node [style=node] (24) at (-3.5, -3) {};
		\node [style=node] (25) at (2.5, 3.5) {};
	%\end{pgfonlayer}
	%\begin{pgfonlayer}{edgelayer}
	
	%\end{pgfonlayer}
\node [setc] at (0,0.5) {};
\node [setc] at (1,1.5) {};
\node [setc] at (2,2.5) {};
\node [setc] at (2.5,2) {};
\node [setc] at (2.5,1) {};
\node [setc] at (0.5,-0.5) {};
\node [setc] at (1.5,-1.5) {};
\node [setc] at (2.5,-1) {};
\node [setc] at (3,0.5) {};
\node [setc] at (1,-0.5) {};
\node [setc] at (1,0.5) {};
\node [setc] at (-1,-0.5) {};
\node [setc] at (0,-1) {};
\node [setc] at (-2,-2) {};
\node [setc] at (-1.5,-2) {};
\node [setc] at (0,-1.5) {};
\node [setc] at (1,-1) {};
\node [setc] at (0,1.5) {};
\node [setc] at (1,2.5) {};
\node [setc] at (0.5,3.5) {};
\node [setc] at (-1.5,0) {};
\node [setc] at (-0.5,0.5) {};
\node [setc] at (-2.5,-1) {};
\node [setc] at (-4,-0.5) {};
\node [setc] at (-4,-2) {};
\node [setc] at (-3,-1.5) {};
\node [setc] at (-3,-0.5) {};
\node [setc] at (-4,1) {};
\node [setc] at (-1,2.5) {};
\node [setc] at (0,2.5) {};
\node [setc] at (-0.5,3) {};
\node [setc] at (1.5,3) {};
\node [setc] at (2.5,3) {};
\node [setc] at (-0.5,1) {};
\node [setc] at (0.5,1) {};
\node [setc] at (-2,-1) {};
\node [setc] at (-1.5,-1) {};
\node [setc] at (-3.5,-2) {};
\node [setc] at (-2.5,-2.5) {};

\filldraw[thick,fill=green,fill opacity=0.4] (20.center) -- (21.center) -- (23.center) -- (22.center) -- cycle;

		%\draw (21) to (20);
		%\draw (23) to (22);
		\draw [ultra thick] (24) to (25);
\end{tikzpicture}}
\caption{Einordnung: mit der richtige Hyperplane}
\end{figure}
\end{frame}

\begin{frame}{\insertsectionhead - \insertsubsectionhead - SVM}
%Bild mit blöden Testdaten (keine richtige Teilung möglich) (F 62)
\begin{figure}
\scalebox{1.0}{\input{img/svm3.tikz}}
\caption{Training: ungünstige Daten}
\end{figure}
\end{frame}

\begin{frame}{\insertsectionhead - \insertsubsectionhead - SVM\cite{dwh}}
Mehrere Klassen:
\begin{itemize}
\item One-versus-all
\item One-versus-one
\end{itemize}

Overfitting
\begin{itemize}
\item Zu viele Trainingsdaten für eine Eigenschaft
\item Lösungen
\begin{itemize}
\item Cross-validation
\item Regularization
\end{itemize} 
\end{itemize}
\end{frame}

\begin{frame}{\insertsectionhead - \insertsubsectionhead - SVM}
%Bild mit Overfitting (F 65)
\begin{figure}
\scalebox{1.1}{\tikzstyle{none}=[inner sep=0pt]
\definecolor{hexcolor0x0074ff}{rgb}{0.000,0.455,1.000}
\definecolor{hexcolor0xff2615}{rgb}{1.000,0.149,0.082}

\definecolor{myblack}{rgb}{0.000,0.000,0.000}
\definecolor{mywhite}{rgb}{1.000,1.000,1.000}

\tikzstyle{setA}=[circle,fill=hexcolor0x0074ff,draw=myblack]
\tikzstyle{setB}=[circle,fill=hexcolor0xff2615,draw=myblack]
\tikzstyle{node}=[circle,fill=mywhite,draw=myblack,scale=.1]


\begin{tikzpicture}
	%\begin{pgfonlayer}{nodelayer}
		\node [style=setA] (0) at (-1.75, 1.75) {};
		\node [style=setA] (1) at (-2.5,0.5) {};
		\node [style=setA] (2) at (0,2) {};
		\node [style=setA] (3) at (-1.25, 0.75) {};
		\node [style=setA] (4) at (-2.25, 1) {};
		\node [style=setA] (5) at (-1,0) {};
		\node [style=setA] (6) at (0.5,1.5) {};
		\node [style=setA] (7) at (-1, 1.5) {};
		\node [style=setB] (8) at (1,2) {};
		\node [style=setB] (9) at (0,1) {};
		\node [style=setB] (10) at (0,0.5) {};
		\node [style=setB] (11) at (0,-0.5) {};
		\node [style=setB] (12) at (1,2.5) {};
		\node [style=setB] (13) at (0.5,1) {};
		\node [style=setB] (14) at (-0.5,-0.5) {};
		\node [style=setB] (15) at (-1,-1.5) {};
		\node [style=setB] (16) at (-1.5,-0.5) {};
		\node [style=setB] (17) at (-1,0.5) {};
		\node [style=setA] (18) at (-2.25, -0.25) {};
		\node [style=setB] (19) at (1,1.5) {};
		\node [style=node] (22) at (-3,-1.5) {};
		\node [style=node] (23) at (2,3) {};
	%\end{pgfonlayer}
	%\begin{pgfonlayer}{edgelayer}
		\draw [ultra thick] (22) to (23);
	%\end{pgfonlayer}
\end{tikzpicture}}
\caption{Overfitting: zu nahe}
\end{figure}
\end{frame}

%\subsection{Regression~}
%
%\begin{frame}{\insertsectionhead - \insertsubsectionhead - Einleitung\cite{fahrmeir2007regressionsanalyse}}
%
%\end{frame}

\subsection{Assoziation}	
\subsubsection{Apriori}\label{apriori}
\section{Neue Algorithmen~}

%\begin{frame}{Hier steht der Titel der Folie}
%Wir beginnen mit einer Aufzählung
%\begin{itemize}
%  \item Aufzählzeichen werden als Quadrate dargestellt
%  \begin{itemize}
%    \item Unterpunkte ebenfalls
%    \item Allerdings etwas kleiner
%  \end{itemize}
%\end{itemize}
%\end{frame}

%\section{Kapitel 2}
%
%
%\begin{frame}{Itemize-Test}
%  \begin{itemize}
%    \item Lorem ipsum dolor sit amet, consetetur sadipscing elitr, sed diam
%      nonumy eirmod tempor invidunt ut labore et dolore magna aliquyam
%    \item At vero eos et accusam et justo duo dolores et ea rebum.
%      \begin{itemize}
%        \item Stet clita kasd gubergren, no sea takimata sanctus est Lorem ipsum
%          dolor sit amet!
%          \begin{itemize}
%            \item Nam eget dui.
%            \item Maecenas tempus, tellus eget condimentum rhoncus, sem quam
%              semper libero, sit amet adipiscing sem neque sed ipsum.
%          \end{itemize}
%        \item Duis leo
%      \end{itemize}
%    \item Aliquam lorem ante, dapibus in, viverra quis, feugiat a, tellus. 
%  \end{itemize}
%\end{frame}
%
%
%\subsection{Unterkapitel 1}
%
%
%\begin{frame}{Mathe-Test}
%  Gaußsche Summenformel:
%  \[1 + 2 + 3 + 4 + \ldots + n = \sum_{k=1}^n k = \frac{n(n+1)}{2}\]
%  Faltung:
%  \[(f*g)(\xi) := \int_{\mathbb{R}^n} f(y)g(\xi-y)\mathrm{d}y\]
%\end{frame}
%
%
%
%\section{Ende}
%
%
%\begin{frame}{Farbtest}
%  \color{tuRed}
%  Dies ist ein Text in tuRed.
%
%  \color{tuSecondaryDark80}
%  Dies ist ein Text in tuSecondaryDark80.
%
%  \color{tuSecondaryLight}
%  Dies ist ein Text in tuSecondaryLight.
%\end{frame}
%
%
%\begin{frame}{Verwendung von Spalten}
%  \begin{columns}[onlytextwidth]
%    \column{0.5\textwidth}
%      Dies ist die erste Spalte.
%      Die Angabe der Option \texttt{[onlytextwidth]}
%      sorgt dafür, dass die Spaltenbreite korrekt eingehalten wird.
%    \column{0.5\textwidth}
%      Dies ist die zweite Spalte mit weiteren Informationen.
%  \end{columns}
%\end{frame}
%
%
%\begin{frame}{Blöcke}
%  \begin{block}{Diest ist ein Block}
%    Lorem ipsum dolor sit amet, consetetur sadipscing elitr, sed diam
%    nonumy eirmod tempor invidunt ut labore et dolore magna aliquyam
%  \end{block}
%  \begin{exampleblock}{Diest ist ein Example-Block}
%    Lorem ipsum dolor sit amet, consetetur sadipscing elitr, sed diam
%    nonumy eirmod tempor invidunt ut labore et dolore magna aliquyam
%  \end{exampleblock}
%  \begin{alertblock}{Diest ist ein Alert-Block}
%    Lorem ipsum dolor sit amet, consetetur sadipscing elitr, sed diam
%    nonumy eirmod tempor invidunt ut labore et dolore magna aliquyam
%  \end{alertblock}
%\end{frame}
%
%
%\begin{frame}[fragile, allowframebreaks]{Quellcode}
%\begin{lstlisting}[language=Java]
%/**
% * @author Vorname Name 1234567 Gruppe 4a
% * @author Vorname Name 1234567 Gruppe 4a
% *         <p>
% *         Diese Klasse ist fuer Aufgabe XX und macht: Blablabla ...
% *         (Aufgabentext oder kurze Beschreibung der Aufgabe usw. Also was es 
% machen soll)
% */
%public class Uhrzeit {
%    /**
%     * wie immer halt ...
%     *
%     * @param args [0] = die Zeitzone
%     */
%    public static void main(String[] args) {
%        int input;
%        try {   // muesst ihr noch nicht kennen, ist nur zum Abfangen von Fehlerfaellen da
%            input = Integer.parseInt(args[0]); // Einlesen des Argumentes und umwandeln in in
%        } catch (ArrayIndexOutOfBoundsException | NumberFormatException e) {
%            // 1. Fehlerfall das Argument interessiert nicht
%            // 2. das Argument ist keine int Zahl
%            System.err.printf("Rufen sie das Programm auf mit \"java Uhrzeit2 [0-2]\"%n");
%            // Schreibe das auf den Error Outputstream
%            return;
%        }
%
%        if (input < - 12 || input > 12) {
%            System.err.printf("%2d ist keine gueltige Zeitzone.%n", input);
%            return;
%        }
%
%        long t = System.currentTimeMillis();
%
%        // die Stunden addieren
%        t += input * 1000 * 60 * 60;
%
%        // umwandeln in richtige Zeitformate
%        int h = (int) (t / (1000 * 60 * 60)) % 24;
%        int m = (int) (t / (1000 * 60)) % 60;
%        int s = (int) (t / 1000) % 60;
%
%        switch (input) {
%            case 0:
%                System.out.printf("Es ist jetzt %2d:%2d Uhr und %2d Sekunden %%%%%%%%%(UTC).%n", h, m, s);
%                break;
%            case 1:
%                System.out.printf("Es ist jetzt %2d:%2d Uhr und %2d Sekunden %%%%%%%%%(MEZ).%n", h, m, s);
%                break;
%            case 2:
%                System.out.printf("Es ist jetzt %2d:%2d Uhr und %2d Sekunden %%%%%%%%%(MESZ).%n", h, m, s);
%                break;
%            default:
%                System.out.printf("Es ist jetzt %2d:%2d Uhr und %2d %%%%%%%%%Sekunden.%n", h, m, s);
%        }
%    }
%}
%\end{lstlisting}
%\end{frame}
%
%
%\begin{frame}[highlight]{Wichtig}
%Diese Folie ist wichtig!
%\end{frame}

\begin{frame}[highlight]{Diskussion}
\centering
Gibt es Fragen?
\end{frame}

\begin{frame}{Danke}
\centering
Vielen Dank für Ihre Aufmerksamkeit und Ihr Interesse.
\end{frame}

\begin{frame}[allowframebreaks]
\frametitle{Literatur}
%\footnotesize
\bibliographystyle{../IEEEtranBST/IEEEtran} % nummerierung
\setbeamertemplate{bibliography item}{[\theenumiv]}
%\bibliographystyle{apacite}
\bibliography{../lit}
\end{frame}

\end{document}
